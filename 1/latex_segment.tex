\documentclass[12pt,letterpaper]{article}

\author{Jordan Bayles}
\title{Assignment 1: Written Portion}
\date{Monday, 21 January, 2013}

%%Usepackage declarations
\usepackage[left=1in,top=1in,right=1in,bottom=1in]{geometry}
\usepackage{lastpage}
\usepackage{sectsty}
\usepackage{slashed}
\usepackage{amsmath}
\usepackage{amsfonts}
\usepackage{latexsym}
% Include for easy import of full pdf pages
\usepackage{pdfpages}
% Include for use of images
\usepackage{graphicx}
% Include for use of [H] placement specifier
\usepackage{float}
% Include for use of \toprule, \midrule, \bottomrule in tabular env.
\usepackage{booktabs}
% Include for setting spacing between lines
\usepackage{setspace}
% Code listing packages
\usepackage{listings}
\usepackage{xcolor}
\usepackage{color}
\usepackage[font=small,format=plain,labelfont=bf,up,textfont=it,up]{caption}

%% Package usages
\sectionfont{\normalsize}
\subsectionfont{\small}

%% New commands
\newcommand{\comment}[1]{}
\newcommand{\field}[1]{\mathbb{#1}} % requires amsfonts
\newcommand{\script}[1]{\mathcal{#1}} % requires amsfonts
\newcommand{\pd}[2]{\frac{\partial#1}{\partial#2}}

%% Access document variables
\makeatletter
\let\thetitle\@title
\let\theauthor\@author
\let\thedate\@date
\makeatother

%% Color Definitions
\definecolor{dkgreen}{rgb}{0,0.6,0}
\definecolor{gray}{rgb}{0.5,0.5,0.5}
\definecolor{mauve}{rgb}{0.58,0,0.82}
\definecolor{lightgrey}{gray}{0.8}
\definecolor{darkgrey}{gray}{1.6}

%% Code Listing Configuration
\DeclareCaptionFormat{listing}{\colorbox{gray}{\parbox{0.987\linewidth}{#1#2#3}}}
\captionsetup[lstlisting]{format=listing, labelfont=white, indention=0pt, textfont=white, margin=0pt, font={bf,footnotesize}, singlelinecheck=false}
\DeclareCaptionFont{white}{\color{white}}
\renewcommand{\lstlistingname}{Code}
\lstset{ %
  %Some lang opts: C++, C, Java, Python, Matlab, TeX, HTML, SQL, Verilog, VHDL, make, ...
  basicstyle=\footnotesize\ttfamily , % the size of the fonts that are used for the code
  numbers=left,                       % where to put the line-numbers
  numberstyle=\scriptsize\color{darkgray}, % the style that is used for the line-numbers
  stepnumber=2,                       % the step between two line-numbers. 
  numbersep=5pt,                      % how far the line-numbers are from the code
  backgroundcolor=\color{white},      % choose the background color. You must add \usepackage{color}
  showspaces=false,                   % show spaces adding particular underscores
  showstringspaces=false,             % underline spaces within strings
  showtabs=false,                     % show tabs within strings adding particular underscores
  frame=tb,                           % adds a frame around the code
  rulesepcolor=\color{gray},          % if not set, the frame-color may be changed on line-breaks within not-black text (e.g. commens (green here))
  tabsize=2,                          % sets default tabsize to 2 spaces
  captionpos=t,                       % sets the caption-position
  breaklines=true,                    % sets automatic line breaking
  breakatwhitespace=false,            % sets if automatic breaks should only happen at whitespace
  title=\lstname,                     % show the filename of files included with \lstinputlisting;
  keywordstyle=\color{blue},          % keyword style
  commentstyle=\color{dkgreen},       % comment style
  stringstyle=\color{mauve},          % string literal style
  escapeinside={\%*}{*)},             % if you want to add a comment within your code
  morekeywords={*,...}                % if you want to add more keywords to the set
  framexbottommargin=5pt,
}

\begin{document}
\begin{flushright}
\theauthor\\
\thedate
\end{flushright}
\begin{center}
\thetitle
\end{center}
% Include a full PDF:
% \includepdf[pages=-]{written_portion.pdf}
% Or for a separate file:
%\lstinputlisting[language=Python]{source_filename.py}

\section*{Problem 1}
There are many ways to transfer files between a local client and remote server,
and the one that you select depends on the capabilities of the server. Some
common protocols include
SFTP (SSH file transfer protocol), rsync, AFP (Apple Filing Protocol), which
each have their respective strengths, such as ease of use and security.

\section*{Problem 2}
Revision controls systems are primarily a method for tracking changes to
a software project over time. Some common implementations include Git and
Mercurial SCM. Some of the most common uses include controlling who has
access to canon code ("Blessed repository" in Git lingo), "branching" off
the main code in order to develop a new feature, and determining when bugs
were added to the code base (and by who, such as using \verb!git blame!).

\section*{Problem 3}
Redirection is defined as switching a standard stream of data to either a
non-default destination or source. Typically in UNIX systems, the redirection
symbols $\langle$ and $\rangle$ are used to redirect process streams to
or from a file. Piping is actually a special case of redirection, and is the
act of chaining process input and output by using the \verb!|! symbol.

\section*{Problem 4}
Make is a software utility that uses a special filetype known as a ``Makefile''
to turn source code into either program executables or libraries. Although for
some languages the compiler can perform this task and many IDEs (such as
Visual Studio) as well, make is still widespread in use, especially in UNIX
derived systems. Some of the primary advantages of using make include that it
is lightweight and included on many Linux and BSD distributions by default (or
is easy to add), thus programs that depend on make are easy to compile and use.
Make is also excellent for development, because it can partially rebuild a program
when only some of its source files have been edited and make can do more than create
packages: it can install/remove them, clean object files, and generate tag
tables for them, among other things.

\section*{Problem 5}
The makefile is general broken into two portions: macros and rules.
\lstset{caption={Sample macros},label=macro,language=make}
\begin{lstlisting}
PACKAGE         = package_name
CC              = icc
\end{lstlisting}
\emph{Macros} are similar to the C \verb!#define! instruction, and are used to contain
constants such as the compiler name, program version and name, as well as flags
for programs called in the program.
\lstset{caption={Sample rule},label=rule,language=make}
\begin{lstlisting}
all:
        echo "Do nothing by default"
        echo "Try 'make target'"
\end{lstlisting}
\emph{Rules}, by comparison, are a set of instructions that are used to
build a \emph{target} such as ``release'' or ``debug.'' A rule contains all of the
information required to complete a certain task, and can include any command
supported by the operating system, allowing you to perform nearly any task. In
practice, make is typically called with a rule supplied (the above rule
would be run with the command \verb~make all~, for example).

\section*{Problem 6}
\lstset{caption={Find rule},label=find,language=bash}
\begin{lstlisting}
find . -type f -exec file '{}' \;
\end{lstlisting}

Sample output:
\lstset{caption={Sample find output},label=out,language=bash}
\begin{lstlisting}
jordan@fenrir /home/jordan/Dropbox/schoolwork/cs311/1  (master)
-> find . -type f -exec file '{}' \;
./latex_segment.aux: LaTeX table of contents, 
./python_segment.py: Python script, ASCII text executable
./latex_segment.pdf: PDF document, version 1.5
./latex_segment.log: ASCII text
./c_segment.c: C source, ASCII text
./names.txt: ASCII text, with very long lines
./latex_segment.tex: LaTeX 2e document, ASCII text
./words.txt: ASCII text, with very long lines, with no line terminators
\end{lstlisting}

\section*{Problem 7: Sieve of Eratosthenes}
\lstinputlisting[caption={Sieve implementation},language=C]{c_segment.c}
\end{document}
