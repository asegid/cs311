\documentclass[letterpaper,10pt,titlepage]{article}

\usepackage{graphicx}

\usepackage{amssymb}
\usepackage{amsmath}
\usepackage{amsthm}

\usepackage{alltt}
\usepackage{float}
\usepackage{color}
\usepackage{verbatim}

\usepackage{geometry}
\geometry{textheight=10in, textwidth=7.5in}

\PassOptionsToPackage{hyphens}{url}\usepackage{hyperref}

\def\name{Jordan Bayles (baylesj)}
\author{\name}
\title{CS 311 Final Exam: Windows/POSIX API Discussion}

%pull in the necessary preamble matter for pygments output
\input{pygments.tex}

%% The following metadata will show up in the PDF properties
\hypersetup{
  colorlinks = true,
  urlcolor = black,
  pdfauthor = {\name},
  pdfkeywords = {cs311 ``operating systems'' final exam essay},
  pdftitle = {CS 311 Final Exam: Windows/POSIX API Diff},
  pdfsubject = {CS 311 Final Exam},
  pdfpagemode = UseNone
}

\parindent = 0.0 in
\parskip = 0.2 in

% Compare and contrast the Windows and POSIX APIs. Include code is optional
\begin{document}
\maketitle

\section{File Input and Output (I/O)}
The POSIX concept of standard input, output and error file descriptors is roughly
mirrored in Windows: upon program initialization, startup code initializes
\verb!stdin!, \verb!stdout!, and \verb!stderr! streams, which are directed to
console keyboard and screen by default.\cite{unixamg}

The standard I/O library (\verb!stdio.h!) providing system calls is part of
ANSI C, so is available on both UNIX and Windows systems. Therefore file I/O
can be done similarly on both systems. The UNIX Concept of universality of I/O
is somewhat matched in the windows library, as \verb~CreateFile~ and other
functions can be used for many file and I/O devices, such as files, file streams,
directories, physical disks, volumes, console buffers, tape drives, pipes, etc.
\cite{fmfw}

Part of the reason why file operation are similar between the two APIs is that
some of windows features are based on the paradigm popularized by BSD UNIX, such
as Windows Sockets (Winsock).

\section{Threading}
On a abstracted level, thread implementations in the POSIX and Windows APIs
are very similar in nature. Most functions have a 1 to 1 mapping between
the APIs, to the point where posix-like threading has been implemented previously
in windows\footnote{pthreads-win32, available at \url{www.sourceware.org/pthreads-win32/}}.

Although the functionality both APIs expose is similar in nature, the underlying
structure and design principles are extremely different. For example, in Windows,
processes are a container unit for threads (the basic execution unit). In POSIX<
threads are merely a special case of processes, the basic unit. These design
considerations are reflected in their implementations, as threading code
in windows is in-kernel (meaning that thread functions are in themselves system
calls) but a system library (\verb!pthreads.h!) that handles the system call usage
in POSIX systems.\cite{pvsthreads}

Some similarities between the two APIs include (1) a required entry point for each
thread, (2) each thread must be passed a single parameter, (3) the thread function
must return a single value, and (4) local parameters are preferred due to the
requirement of synchronization methods to access global variables (since these
are shared between threads in a single process).

Major differences between the APIs include (1) threads in Win32 cannot return
strings, (2) Win32 threads return DWORD values, (3) Win32 does not support
deferred termination, due to the fact that a \verb!TerminateThread! call
can occur at any point in the code, is not immediate, and is not predictable.
UNIX threads terminate when a safe cancellation point is reached, and thus
deferred termination is more predictable.
\section{Mutual Exclusion and Synchronization}
Why did the multithreaded chicken cross the road?\\
to To other side. get the

Both POSIX and Win32 threads suffer from the issue of resource access not being
guaranteed atomic, and solve this problem in a similar fashion. In comparison
to the sychronization primitives (semaphores, mutexes, and conditional variables)
utilized in POSIX, Win32 threads use a set of synchronization objects in order
to control resource
access. These include (1) events, (2) semaphores, (3) mutexes, and (4) critical
sections. Although both APIs included the use of semaphores and mutexes, signals
in Win32 are handled very differently than the condition variables utilized by
the pthreads library. In POSIX, signals are either caught by waiting threads or
just discarded (although code can be structured to ensure the signals are not lost),
versus events in Win32, which remain signaled once in the signaled state. \cite{pvsthreads}

One of the most notable differences in synchronization between win32 and
pthreads is the addition of the \emph{critical section object} in the win32
API. The critical section is similar in principle to mutexes, however it
can not be shared between multiple processes and is more efficient than win32
mutexes. The \verb!EnterCriticalSection!, \verb!TryEnterCriticalSection!, and
\verb!LeaveCriticalSection! functions are utilized with the \verb!CRITICAL_SECTION!
variable type in order to manage critical sections of code. \cite{msthreads}

%\section{Sockets}

\bibliographystyle{plain}
\bibliography{biblio.bib}
\end{document}
